%%
%%  $Header: /u/gcmpack/MITgcm/pkg/icefront/icefront_description.tex,v 1.2 2012/03/08 18:41:06 yunx Exp $
%%  $Name: checkpoint65t $
%%

Package ``ICEFRONT'' is an extension of pkg/shelfice that deals with vertical
ice fronts for application to tidewater glaciers.
(Xu et al., Numerical experiments on subaqueous melting of Greenland tidewater 
 glaciers in response to ocean warming and enhanced subglacial discharge. 
 Annals of Glaciology, 2012. http://ecco2.org/manuscripts/2012/Xu2012.pdf)

It includes

1) a routine (icefront_init_fixed.F), which loads two 2-D variables:
  a) R_icefront: depth of the submerged vertical ice wall in or at the edge of 
     each wet grid of points
  b) icefrontlength: ratio of the horizontal length of vertical ice front to 
     horizontal grid area for points in a)

2) a routine (icefront_thermodynamics.F) similar to shelfice_thermodynamics 
   which applies the ice shelf melt/freeze equations to horizontal locations 
   with vertical ice wall, i.e., containing non-zero values for 1a) and 1b). 

3) a routine (icefront_tendency_apply.F) which applies the forcing computed
   in 2) to grids of ice wall.

The package can be used in conjunction with pkg/shelfice, if both the
horizontal and vertical termini of glaciers need to be represented, or it can
be used on its own, if the glaciers terminate and calve near the grounding
line, i.e., if they have no significant ice shelf cavity beneath them.

Pkg/icefront is separated from pkg/shelfice because

(i) rather than search 2D space for ice shelf cavities, the icefront package
needs to search a 3D space for the ice fronts,

(ii) each grid box can have up to three separate contributions from ice
fronts while ice shelf cavities only have one contribution from the top, and

(iii) eventually the physics of the two packages may diverge to include
different processes, for example, a fancier boundary layer scheme for shelfice
and a calving scheme for icefront.

Pkg/icefront is mostly designed to represent Greenland or Alaska tidewater 
glaciers in model configurations that can resolve the fjords and glacier outlets,
i.e., that have horizontal grid spacing of O(1km) or less.

Verification experiment for pkg/icefront is verification/isomip/input.icefront/
